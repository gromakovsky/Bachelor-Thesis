\documentclass[mathserif]{beamer}

\usetheme{Berkeley}
\usecolortheme{orchid}

\usepackage[utf8x]{inputenc}
\usepackage[english, russian]{babel}
\usepackage{color}
\usepackage{graphicx}

\def\putimg<#1>#2{ \includegraphics<#1>[width=\textwidth]{images/#2} }

\title{Поиск семейств оптимальных маршрутов на морских картах}
\author{Иван Громаковский \\
Научный руководитель: А. С. Ковалев}
\institute{Санкт-Петербургский национальный исследовательский университет \\ информационных технологий, механики и оптики}
\date{13 мая 2015}

\begin{document}

\frame{\titlepage}
\note{Здравствуйте!}

\begin{frame}
    \frametitle{Цели работы}
    \begin{itemize}
        \item Формализация понятия оптимальности
        \item Поиск семейств оптимальных маршрутов для принятия
          решения пользователем
        \item Реализация алгоритма, работающая в режиме реального времени
    \end{itemize}
\end{frame}

\begin{frame}
    \frametitle{Актуальность задачи}
    Проблемы поиска единственного маршрута:
    \begin{itemize}
        \item Критерии оптимальности не всегда очевидны и формализуемы
        \item Не оставляет выбора пользователю
        \item «Пробки»
        \item TODO: что-то ещё на этот слайд?
    \end{itemize}
\end{frame}
        
\begin{frame}
    \frametitle{Оптимальность маршрутов}
    Под семейством оптимальных маршрутов из точки A в точку B будем
    понимать максимальное по включению множество маршрутов со следующими свойствами:
    \begin{itemize}
        \item Для любых двух маршрутов найдётся препятствие, размеры
          которого сопоставимы с длиной кратчайшего из маршрутов,
          которое обходится с разных сторон
        \item TODO: получше сформулировать первый пункт
        \item TODO: ещё пункты?
    \end{itemize}
\end{frame}

\begin{frame}
    \frametitle{Существующие алгоритмы}
    Известные алгоритмы множественного поиска путей в графе:
    \begin{itemize}
        \item<1-1> разрабатывались для дргих целей
        \item<1-1> не учитывают топологию
        \item<1-1> строят очень похожие маршруты на морских картах
    \end{itemize}
    \putimg<2-2>{comparison-with-existing-bad.png} 
    \putimg<3-3>{comparison-with-existing-good.png} 

\end{frame}

\begin{frame}
    \frametitle{Предобработка данных}
    \begin{itemize}
        \item Карта → полигон
        \item Оффсет (по-русски?) полигона
        \item Граф по сетке на плоскости + граф локальной видимости
        \item Ограничение длины ребра
        \item Дополнительные рёбра
    \end{itemize}
\end{frame}

\begin{frame}
    \frametitle{Поиск одного маршрута}
    \begin{itemize}
        \item Добавление вершин в граф
        \item Алгоритм Дейкстры
        \item Сокращение маршрута
        \begin{itemize}
            \item Если подпуть A → B → C можно выгодно заменить на A → C, заменяем 
            \item Подразбиение маршрута
        \end{itemize}
        \item Сглаживание маршрута
        \item TODO: картинка (картинки) для демонстрации
    \end{itemize}
\end{frame}

\begin{frame}
    \frametitle{Поиск нескольких маршрутов}
    \begin{itemize}
        \item Поиск одного маршрута
        \item Обновление весов:
        \begin{itemize}
            \item Потенциалы как функция кратчайших расстояний от фиктивной вершины 
            \item Обновление потенциалов
            \item Применение потенциалов
        \end{itemize}
        \item Проверка критерия остановки:
        \begin{itemize}
            \item Длина маршрута
            \item Метрики на маршрутах
        \end{itemize}
    \end{itemize}
\end{frame}

\begin{frame}
    \frametitle{Обновление весов}
    \begin{itemize}
        \item Потенциалы должны быть множителями, а не слагаемымми
        \item TODO: Картинка
        \item На маршруте потенциалы меньше, чем поблизости
        \item TODO: Картинка
    \end{itemize}
\end{frame}

\begin{frame}
    \frametitle{Метрики}
    \begin{itemize}
        \item $\rho_1 (P, Q) = max_{u \in P} min_{v \in Q} \rho_g(u, v)$
        \item TODO: картинка для демонстрации
        \item TODO: вторая метрика
        \item TODO: картинка для демонстрации
    \end{itemize}
\end{frame}

\begin{frame}
    \frametitle{Вычисление метрик}
    \begin{itemize}
        \item Фиктивная вершина
        \item Обход Дейкстры, пока не посещены вершины второго пути 
        \item Заканчиваем, если достигли необходимого значения
        \item Поиск ближайшей вершины (вторая метрика) — перебором
        \item На практике затраты на вторую метрику такие же низкие
    \end{itemize}
\end{frame}

\begin{frame}
    \frametitle{Результаты}
    \begin{itemize}
        \item<1-1> Разработан и реализован алгоритм построения семейств оптимальных маршрутов 
        \item<1-1> Проведено сравнение с существующими подходами 
        \item<1-1> Поиск выполняется примерно за полсекунды 
    \end{itemize}
    \putimg<2-2>{results.png}
\end{frame}

\end{document}

