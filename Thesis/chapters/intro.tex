% -*-coding: utf-8-*-
\startprefacepage

Задача поиска кратчайшего пути между двумя точками при наличии
полигональных препятствий является одной из базовых задач
вычислительной геометрии (computational
geometry) \cite{PrSh}. Для её точного решения строится граф
видимости (visibility graph) \cite{deBerg}, а затем выполняется поиск
кратчайшего пути в графе, например, с помощью алгоритма Дейкстры
\cite{Cormen}. Однако в графе видимости может быть слишком много
рёбер, из-за чего алгоритм Дейкстры будет работать неприемлимо долго.
В таком случае строится некоторый навигационный граф, рёбра которого
не пересекают препятствия, и затем выполняется поиск кратчайшего пути
в этом графе. В результате получается путь, близкий к кратчайшему.
Решая данную задачу, можно находить маршрут для корабля, представляя
сушу как полигональные препятствия. 

Однако не всегда достаточно предъявить лишь один маршрут. В некоторых
ситуациях кратчайший маршрут (по какой-либо фиксированной метрике) не
будет оптимален ввиду других, трудно описываемых критериев. Например,
где-то может быть нежелательно плавать по экономическим или
политическим причинам. И если первый фактор ещё может быть как-то
формализован, то говорить о каком-либо математическом
описании политических причин едва ли представляется возможным. Отсюда
возникает другая задача: найти семейство оптимальных маршрутов между
двумя точками, чтобы пользователь (капитан корабля) сам мог принять
решение о том, как нужно плыть. Возникает вопрос: что такое семейство
оптимальных маршрутов между двумя точками?

В данной работе произведена попытка формализовать понятие семейства
оптимальных маршрутов. Также произведено исследование существующих
алгоритмов поиска нескольких путей в графе применительно к морским
картам. Предложен и реализован алгоритм поиска семейств оптимальных маршрутов.

В первой главе приведён более подробный обзор предметной области и
более чёткая формулировка задачи. Во второй главе описаны некоторые
известные алгоритмы множественного поиска путей в графе, приведено
исследование их применимости к поставленной задаче. Дальше ещё что-то будет.

\FloatBarrier

