%-*-coding: utf-8-*-
\chapter{Обзор предметной области}

\FloatBarrier
\section{Планирование путей}

\emph{Планирование путей (англ. path planning, motion planning)} ---
область computer science, решающая задачу поиска пути из одной точки в
другую, удовлетворяющего некоторым ограничениям. В основе
планирования путей лежат такие науки, как вычислительная геометрия
и теория графов. В данной работе рассматривается планирование
маршрутов по воде по всему миру. В этом случае суша представляет из
себя полигональные препятствия. Как известно (TODO: сослаться), любой
кратчайший путь между двумя вершинами при наличии полигональных препятствий
представляет собой ломаную, вершины которой --- вершины полигонов. В
дальнейшем в данной работе слова «\emph{маршрут}» и «\emph{путь}»
будут использованы взаимозаменяемо.

\emph{Граф видимости (англ. visibility graph)} --- граф, вершины
которого являются вершинами полигональных препятствий, а ребро между
вершинами $u$ и $v$ принадлежит графу тогда и только тогда, когда
отрезок $uv$ не пересекает ни одно препятствие.

Таким образом, любой кратчайший путь между двумя точками является
кратчайшим путём в графе видимости и может быть найден, например, с
помощью алгоритма Дейкстры. Однако граф видимости, построенный по
контурам суши со всего мира, взятым с мелкомасштабной карты, содержит
слишком много рёбер, из-за чего поиск может выполняться довольно
длительное время. В связи с этим возникает другая задача: найти
маршрут, близкий к кратчайшему. Для этой задачи необязательно строить
граф видимости, достаточно ограничиться навигационным графом.

\emph{Навигационный граф} --- граф, построенный по полигональным
препятствиям, вершины которого соответствуют некоторым точкам в
пространстве, а ребро между вершинами $u$ и $v$ может принадлежать графу только
тогда, когда отрезок $uv$ не пересекает ни одно препятствие.

\FloatBarrier
\section{Планирование мультипутей}

\emph{Планирование мультипутей (англ. multipath planning)} ---
область computer science, решающая задачу поиска нескольких путей из
одной точки в другую, таких что каждый путь удовлетворяет некоторым
ограничениям, и при этом на множеством путей также наложены
ограничения. Например, задача поиска $k$ кратчайших путей через
полигональные препятствия.

При поиске единственного маршрута (например, кратчайшего по какой-либо
метрике) могут возникнуть следующие проблемы:
\begin{enumerate}
    \item Критерии оптимальности не всегда очевидны и формализуемы.
      Чаще всего интересует кратчайший маршрут, но иногда может быть
      нужен, например, самый экономный маршрут, который не обязан
      совпадать с кратчайшим. Также могут быть и другие критерии, в
      том числе составные, которые едва ли представляется возможным
      описать какой-либо метрикой. Например, вряд ли возможно
      формализовать политические факторы или личные предпочтения пользователя.
    \item Ненадёжность. Возможна ситуация, когда находится
      единственный маршрут, но воспользоваться им не представляется
      возможным. Например, в каком-то месте может быть временный
      запрет или просто слишком мелко. В таком случае требуется
      запасной вариант.
    \item Повышенная загруженность. Если приложение, находящее ровно
      один маршрут, набирает популярность, это может привести к тому,
      что некоторыми маршрутами будут очень активно пользоваться, что
      приведёт к повышенной загруженности. Например, одна компания
      может направить большое число судов из одной точки в другую.
      Если все они пойдут через один и тот же канал, то могут потерять
      крайне много времени.
\end{enumerate}

Известны различные алгоритмы поиска нескольких маршрутов, более
подробный обзор которых будет приведён во второй главе.

\FloatBarrier
\section{Постановка задачи}

В данной работе требуется исследовать вопросы планирования
мультипутей на морских картах. Для этого прежде всего требуется
наложить ограничения на множество находимых маршрутов, тем самым
определив понятие семейства оптимальных маршрутов. Следующий этап
состоит в исследовании имеющихся алгоритмов планирования мультипутей
для выяснения их применимости к поиску маршрутов по воде. Затем
требуется разработать и реализовать новый алгоритм, находящий
семейство оптимальных маршрутов. Необходимо, чтобы алгоритм работал в
режиме реального времени, затрачивая меньше секунды на запрос.

\FloatBarrier
\section{Семейство оптимальных маршрутов}

При поиске нескольких маршрутов в первую очередь возникает вопрос о
том, каким условиям должно удовлетворять результирующее множество
маршрутов. Разумеется, маршруты не должны быть похожи. TODO

\FloatBarrier

