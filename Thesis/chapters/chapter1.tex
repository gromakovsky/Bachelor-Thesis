%-*-coding: utf-8-*-
\chapter{Обзор предметной области}

\section{Планирование маршрутов}

Задача планирования маршрута (англ. path planning) заключается в
поиске пути из точки $A$ в точку $B$, удовлетворяющего некоторым
ограничениям. В данной работе слова «\emph{маршрут}» и «\emph{путь}»
используются взаимозаменяемо. При планировании маршрута по воде
ограничениями являются полигональные препятствия (материки, острова и
так далее), которые нельзя пересекать. Один из способов решения этой
задачи состоит в построении графа, вершины которого соответствуют
некоторым точкам в мире, а ребро между двумя вершинами может
присутствовать в графе только при условии, что оно не пересекает ни
одно из препятствий. Известно~\cite{visibility-proof}, что любой кратчайший путь
между двумя вершинами представляет из себя ломаную, вершины которой
являются вершинами препятствий. Потому для поиска кратчайшего пути
требуется построить \emph{граф видимости}.

\emph{Граф видимости (англ. visibility graph)} --- граф, вершины
которого являются вершинами полигональных препятствий, а ребро между
вершинами $u$ и $v$ принадлежит графу тогда и только тогда, когда
отрезок $uv$ не пересекает ни одно препятствие.

Таким образом, любой кратчайший путь между двумя точками является
кратчайшим путём в графе видимости и может быть найден с помощью
любого алгоритма поиска кратчайшего пути в графе. Одними из наиболее
популярных алгоритмов являются алгоритм
Дейкстры~\cite{dijkstra1959note} и $A^*$~\cite{hart1968formal}. Также
стоит отметить, что довольно часто не требуется находить кратчайший
маршрут, достаточно найти какой-нибудь корректный маршрут, желательно
близкий к кратчайшему. В этом случае используется \emph{навигационный
  граф}.

\emph{Навигационный граф} --- граф, построенный по полигональным
препятствиям, вершины которого соответствуют некоторым точкам в
пространстве, а ребро между вершинами $u$ и $v$ может принадлежать
графу только тогда, когда отрезок $uv$ не пересекает ни одно
препятствие.

Как было сказано во введении, задача планирования маршрута имеет
множество недостатков при поиске маршрутов по воде. Поэтому возникает
задача поиска семейств маршрутов. Данная задача исследована не так
хорошо, как планирование единственного маршрута, однако известны
различные алгоритмы, обзор которых приведён в следующем разделе. При
поиске семейств маршрутов по воде естественно требовать, чтобы все
маршруты были, неформально говоря, попарно непохожи. Например, на
рисунке \ref{fig:similar-paths} маршруты не имеют общих рёбер, но
являются очень похожими. В то же время на рисунке
\ref{fig:dissimilar-paths} представлены непохожие маршруты, поскольку
то, с какой стороны обходится остров, может быть достаточно
существенно. Более наглядно неформальное описание похожести маршрутов
проиллюстрировано на рисунках \ref{fig:similar-paths2} и \ref
{fig:dissimilar-paths2}.

\begin{figure}
    \centering
    \begin{minipage}{.5\textwidth}
        \centering
        \includegraphics[width=\textwidth, clip=true, trim = 300 0 300
        0]{Introduction/similar}
        \captionof{figure}{Похожие маршруты}
        \label{fig:similar-paths}
    \end{minipage}%
    \begin{minipage}{.5\textwidth}
        \centering
        \includegraphics[width=\textwidth, clip=true, trim = 300 0 300
        0]{Introduction/dissimilar}
        \captionof{figure}{Непохожие маршруты}
        \label{fig:dissimilar-paths}
    \end{minipage}
\end{figure}

\begin{figure}
    \includegraphics[width=\textwidth]{Introduction/similar2}
    \caption{Похожие маршруты}
    \label{fig:similar-paths2}
\end{figure}

\begin{figure}
    \includegraphics[width=\textwidth]{Introduction/dissimilar2}
    \caption{Непохожие маршруты}
    \label{fig:dissimilar-paths2}
\end{figure}

\FloatBarrier

\section{Существующие алгоритмы поиска семейств маршрутов}

Известно множество подходов к проблеме поиска семейств
маршрутов~\cite{lim2005shortest, dial1971probabilistic, mafast}.
Однако они обладают следующими недостатками:
  
\begin{itemize}
    \item Все решения разрабатывались для других целей. Например, для
      поиска маршрутов по дорогам или в транспортных сетях. В этих
      случаях имеется граф дорог, в котором осуществляется поиск путей.
      Если два пути имеют мало общих рёбер, то они, как правило, имеют
      существенные различия. В то же время, два маршрута по воде, не
      имеющие общих рёбер, могут проходить по одним и тем же морям и
      каналам.
    \item Предложенные алгоритмы работают с довольно абстрактными
      графами и не учитывают привязанность вершин к реальным координатам
      в мире.
    \item Следствием первых двух пунктов является то, что при
      применении таких решений к задаче поиска семейств маршрутов по
      воде находятся слишком похожие маршруты.
\end{itemize}

\FloatBarrier

\subsection{Lim, Kim, 2005}

\FloatBarrier

\subsection{Dial, 1970}

\FloatBarrier

\subsection{Dijkstra-Hyperstar}

\FloatBarrier

\section{Формальная постановка задачи}

На основе проведённого обзора и неформальных требований можно
сформулировать формальную постановку задачи. Дано множество
полигональных препятствий $C$, начальная точка $S$ и конечная точка
$D$. По полигональным препятствиям строится навигационный граф.
Требуется найти множество путей $P$ из точки $S$ в точку $D$,
удовлетворяющих следующим свойствам:
\begin{itemize}
  \item Каждый путь $p \in P$ представлен ломаной, любой отрезок
    которой не пересекает ни один контур из $C$.
  \item Пусть $q$ --- кратчайший путь, а $len(p)$ --- длина пути $p$.
    Тогда $\forall p \in P: len(p) < C_0 \cdot len(q)$, где $C_0 > 1$
    является параметром задачи.
  \item Любыe два пути $p, q \in P$ удовлетворяют критерию
    непохожести, сформулированному ниже.
\end{itemize}

Для определения похожести маршрутов используются две метрики на
маршрутах:
\begin{equation*}
    \rho_1 (P, Q) = \max(\max_{u \in P} \min_{v \in Q} \rho_g(u,
    v), \max_{u \in Q} \min_{v \in P} \rho_g(u, v))
\end{equation*}

$\rho_g(u, v)$ означает длину кратчайшего пути в навигационном графе
между вершинами $u$ и $v$. В данной метрике для каждой вершины
маршрута рассматривается длина кратчайшего пути до второго маршрута.
Значением метрики является максимум всех таких длин. Большое значение
этой метрики означает, что для какой-то вершины кратчайший путь в
навигационном графе до второго маршрута имеет большую длину, что
является хорошим показателем того, что маршруты непохожи.
\begin{equation*}
    \rho_2 (P, Q) = \max(\max_{u \in P} \frac{\min\limits_{v \in Q_u}
    \rho_g(u, v)}{\min\limits_{v \in Q_u} \rho_r(u, v)}, \max\limits_{u \in Q} \frac{\min\limits_{v \in P_u}
    \rho_g(u, v)}{\min\limits_{v \in P_u} \rho_r(u, v)})
\end{equation*}
\begin{equation*}
    P_v = \{ u \in P : \rho_r(u, v) > \varepsilon \cdot len(P) \}
\end{equation*}
\begin{equation*}
    Q_u = \{ v \in Q : \rho_r(u, v) > \varepsilon \cdot len(Q) \}
\end{equation*}

$\rho_r(u, v)$ означает расстояние между $u$ и $v$ в мире. Данная
метрика аналогична первой, однако максимум берётся не по длинам
кратчайших расстояний в графе, а по отношению длины кратчайшего
расстояния в графе к расстоянию до ближайшей вершины в мире. Большое
значение этой метрики означает, что для какой-то вершины длина кратчайшего
пути в навигационном графе до второго маршрута существенно больше, чем
кратчайшее расстояние в мире. Это является показателем того, что между
маршрутами имеется препятствие. При этом не рассматриваются пары
слишком близких вершин, поскольку между ними может быть препятствие
очень небольших размеров, которое не делает маршруты непохожими,
однако делает значение метрики большим. Число $\varepsilon$ также
является параметром алгоритма.

На основании этих двух метрик формируется следующий критерий
непохожести маршрутов $P$ и $Q$.
\begin{itemize}
  \item Если $\rho_1(P, Q) > C_{1max} \cdot \min(len(P), len(Q))$, то
    маршруты являются непохожими.
  \item Если $\rho_1(P, Q) < C_{1min} \cdot \min(len(P), len(Q))$, то
    маршруты являются похожими.
  \item Если ни один из первых двух пунктов не выполнен и $\rho_2(P,
    Q)$ > $C_2 \cdot \min(len(P), len(Q))$, то маршруты являются непохожими.

  \item В противном случае маршруты являются похожими.
\end{itemize}

Константы $C_{1max}$, $C_{1min}$, $C_2$ также являются параметрами. В
работе использовались следующие значения параметров:

$C_0 = 2,5$; $C_{1max} = 0,5$; $C_{1min} = 0,2$; $C_2 = 1,15$;
$\varepsilon = 0,1$.

\begin{figure}
    \includegraphics[width=\textwidth]{Solution/metrics/1-dissimilar}
    \caption{Маршруты непохожи по первой метрике}
    \label{fig:1-dissimilar}
\end{figure}

На рисунке~\ref{fig:1-dissimilar} изображены два маршрута (синий и
красный). Чёрная пунктирная линия соединяет две вершины, на которых
достигается значение первой метрики (339 км). При этом длина красного
маршрута составляет 435 км. Поскольку $\frac{339}{435} = 0,78 > 0,5$,
то маршруты считаются непохожими.

\begin{figure}
    \includegraphics[width=\textwidth]{Solution/metrics/1-uncertain-2-similar}
    \caption{Маршруты непохожи по второй метрике}
    \label{fig:1-uncertain-2-similar}
\end{figure}

Для маршрутов, изображённых на
рисунке~\ref{fig:1-uncertain-2-similar}, значение первой метрики равно
502 км, а длина кратчайшего из двух маршрутов равна 1704 км. $0,2 <
\frac{502}{1704} = 0,29 < 0,5$, поэтому используется значение второй метрики.
Поскольку между каждой парой вершин на маршрутах не имеется
препятствий, то значение этой метрики равно 1. Таким образом, эти два
маршрута считаются похожими.

\begin{figure}
    \centering
    \begin{minipage}{.5\textwidth}
        \centering
        \includegraphics[width=\textwidth, clip=true, trim = 300 0 300
        0]{Solution/metrics/1-uncertain-2-dissimilar-gclosest}
        \captionof{figure}{Маршруты «возможно, похожи» по первой метрике\dots}
        \label{fig:1-uncertain-2-dissimilar-gclosest}
    \end{minipage}%
    \begin{minipage}{.5\textwidth}
        \centering
        \includegraphics[width=\textwidth, clip=true, trim = 300 0 300
        0]{Solution/metrics/1-uncertain-2-dissimilar-closest}
        \captionof{figure}{\dots и точно непохожи по второй из-за
          острова между ними.}
        \label{fig:1-uncertain-2-dissimilar-closest}
    \end{minipage}
\end{figure}

Для маршрутов, изображённых на рисунках
\ref{fig:1-uncertain-2-dissimilar-gclosest} и
\ref{fig:1-uncertain-2-dissimilar-closest}, значение первой метрики
равно 307 км (на первом рисунке пунктиром показано, для каких двух
точек оно достигается). Длина кратчайшего из двух маршрутов равна 814
км. $0,2 < \frac{307}{814} = 0,38 < 0,5$, поэтому в данном случае тоже
используется вторая метрика. На втором рисунке пунктиром показана
ближайшая вершина второго пути для той же вершины первого. Расстояние
до неё составляет 251 км. Значение второй метрики достигается между
теми же двумя вершинами и равно $\frac{307}{251} = 1,22$. Поэтому
данные два маршрута считаются непохожими, что полностью согласуется с
неформальными требованиями, описанными ранее.

\begin{figure}
    \includegraphics[width=\textwidth]{Solution/metrics/1-similar}
    \caption{Маршруты похожи по первой метрике}
    \label{fig:1-similar}
\end{figure}

Наконец, на рисунке~\ref{fig:1-similar} представлены маршруты, для
которых значение первой метрики составляет 641 км, а длина кратчайшего
из них --- 4059 км. Таким образом, поскольку $\frac{641}{4059} = 0,16
< 0,2$, то эти маршруты считаются похожими по первой метрике. Хоть
между ними и имеется препятствие, его размеры пренебрежимо малы,
поэтому логично считать такие маршруты похожими.



\FloatBarrier

