%-*-coding: utf-8-*-
\chapter{Обзор предметной области}

\FloatBarrier
\section{Планирование маршрутов}

\emph{Планирование маршрутов (англ. path planning, motion planning)} ---
область computer science, решающая задачу поиска пути из одной точки в
другую, удовлетворяющего некоторым ограничениям. В основе
планирования маршрутов лежат такие науки, как вычислительная геометрия
и теория графов. В данной работе рассматривается планирование
маршрутов по воде по всему миру. В этом случае суша представляет из
себя полигональные препятствия. Как известно (TODO: сослаться), любой
кратчайший путь между двумя вершинами при наличии полигональных препятствий
представляет собой ломаную, вершины которой --- вершины полигонов. 

\emph{Граф видимости (англ. visibility graph)} --- граф, вершины
которого являются вершинами полигональных препятствий, а ребро между
вершинами $u$ и $v$ принадлежит графу тогда и только тогда, когда
отрезок $uv$ не пересекает ни одно препятствие.

Таким образом, любой кратчайший путь между двумя точками является
кратчайшим путём в графе видимости и может быть найден, например, с
помощью алгоритма Дейкстры. Однако граф видимости, построенный по
контурам суши со всего мира, взятым с мелкомасштабной карты, содержит
слишком много рёбер, из-за чего поиск может выполняться довольно
длительное время. В связи с этим возникает другая задача: найти
маршрут, близкий к кратчайшему. Для этой задачи необязательно строить
граф видимости, достаточно ограничиться навигационным графом.

\emph{Навигационный граф} --- граф, построенный по полигональным
препятствиям, вершины которого соответствуют некоторым точкам в
пространстве, а ребро между вершинами $u$ и $v$ может принадлежать графу только
тогда, когда отрезок $uv$ не пересекает ни одно препятствие.

\section{Планирование мультимаршрутов}

TODO


