% -*-coding: utf-8-*-
% This is an AMS-LaTeX v. 1.2 File.

\documentclass{report}

\usepackage[T2A]{fontenc}
\usepackage[utf8x]{inputenc}
\usepackage[english,russian]{babel}
\usepackage{expdlist}
\usepackage[pdftex]{graphicx}
\usepackage{amsmath}
\usepackage{amssymb}
\usepackage{amsthm}
\usepackage{amsfonts}
\usepackage{amsxtra} 
\usepackage{sty/dbl12}
\usepackage{srcltx}
\usepackage{epsfig}
\usepackage{verbatim}
\usepackage{sty/rac}
\usepackage{listings}
\usepackage{placeins}
\usepackage{caption}
\captionsetup[table]{position=t,justification=raggedright,slc=off}

\hoffset = -10mm
\voffset = -20mm
\textheight = 230mm
\textwidth = 165mm

%%%%%%%%%%%%%%%%%%%%%%%%%%%%%%%%%%%%%%%%%%%%%%%%%%%%%%%%%%%%%%%%%%%%%%%%%%%%%%

% Redefine margins and other page formatting

%\setlength{\oddsidemargin}{0.5in}

% Various theorem environments. All of the following have the same numbering
% system as theorem.

\theoremstyle{plain}
\newtheorem{theorem}{Теорема}
\newtheorem{prop}[theorem]{Утверждение}
\newtheorem{corollary}[theorem]{Следствие}
\newtheorem{lemma}[theorem]{Лемма}
\newtheorem{question}[theorem]{Вопрос}
\newtheorem{conjecture}[theorem]{Гипотеза}
\newtheorem{assumption}[theorem]{Предположение}

\theoremstyle{definition}
\newtheorem{definition}[theorem]{Определение}
\newtheorem{notation}[theorem]{Обозначение}
\newtheorem{condition}[theorem]{Условие}
\newtheorem{example}[theorem]{Пример}
\newtheorem{algorithm}[theorem]{Алгоритм}
%\newtheorem{introduction}[theorem]{Introduction}

\renewcommand{\proof}{\\\textbf{Доказательство.}~}

%\def\startprog{\begin{lstlisting}[language=Java,basicstyle=\normalsize\ttfamily]}

%\theoremstyle{remark}
%\newtheorem{remark}[theorem]{Remark}
%\include{header}
%%%%%%%%%%%%%%%%%%%%%%%%%%%%%%%%%%%%%%%%%%%%%%%%%%%%%%%%%%%%%%%%%%%%%%%%%%%%%%%

\numberwithin{theorem}{chapter}        % Numbers theorems "x.y" where x
                                        % is the section number, y is the
                                        % theorem number

%\renewcommand{\thetheorem}{\arabic{chapter}.\arabic{theorem}}

%\makeatletter                          % This sequence of commands will
%\let\c@equation\c@theorem              % incorporate equation numbering
%\makeatother                           % into the theorem numbering scheme

%\renewcommand{\theenumi}{(\roman{enumi})}

%%%%%%%%%%%%%%%%%%%%%%%%%%%%%%%%%%%%%%%%%%%%%%%%%%%%%%%%%%%%%%%%%%%%%%%%%%%%%%


%%%%%%%%%%%%%%%%%%%%%%%%%%%%%%%%%%%%%%%%%%%%%%%%%%%%%%%%%%%%%%%%%%%%%%%%%%%%%%%

%This command creates a box marked ``To Do'' around text.
%To use type \todo{  insert text here  }.

\newcommand{\todo}[1]{\vspace{5 mm}\par \noindent
\marginpar{\textsc{ToDo}}
\framebox{\begin{minipage}[c]{0.95 \textwidth}
\tt #1 \end{minipage}}\vspace{5 mm}\par}

%%%%%%%%%%%%%%%%%%%%%%%%%%%%%%%%%%%%%%%%%%%%%%%%%%%%%%%%%%%%%%%%%%%%%%%%%%%%%%%

\binoppenalty=10000
\relpenalty=10000

\begin{document}


% Begin the front matter as required by Rackham dissertation guidelines

\initializefrontsections

\pagestyle{title}

\begin{center}
Санкт-Петербургский национальный исследовательский университет \\ информационных технологий, механики и оптики

\vspace{2cm}

Кафедра компьютерных технологий

\vspace{3cm}

{\Large И. Е. Громаковский}

\vspace{2cm}

\vbox{\LARGE\bfseries
Построение семейств оптимальных маршрутов \\ на морских картах}

\vspace{4cm}

Бакалаврская работа 

\vspace{1cm}

{\Large Научный руководитель: А. С. Ковалев}

\vspace{5cm}

Санкт-Петербург\\ 2015
\end{center}

\newpage

\setcounter{page}{2}
\pagestyle{plain}

%\dedicationpage{Put a dedication here}
% Dedication page

%\startacknowledgementspage
% Acknowledgements page
%{Put Acknowledgements here}

% Table of contents, list of figures, etc.
\tableofcontents
%\listoffigures


\def\t#1{\mbox{\texttt{\hbox{#1}}}}
\def\b#1{\textbf{#1}}
\def\tb#1{\t{\b{#1}}}

\def\cln#1{\t{#1}}
\def\pcn#1{\t{#1}}
\newcommand{\p}{\par Здесь будет текст...}

\def\drawfigure#1#2#3{
        \begin{figure}[ht]
        \centerline{ \includegraphics[width=8cm]{img/#1}}
        \caption{#2}
        \label{#3}
        \end{figure}
}
\def\drawfigurex#1#2#3#4{
        \begin{figure}[ht]
        \centerline{ \includegraphics[#4]{img/#1}}
        \caption{#2}
        \label{#3}
        \end{figure}
}

% Chapters
\startthechapters
% -*-coding: utf-8-*-
\startprefacepage

Задача поиска кратчайшего пути при наличии полигональных препятствий
хорошо известна и исследована. Классический способ решения этой задачи
описан в~\cite{de2000computational}. Одним из её практических
применений является поиск кратчайшего маршрута кораблей на морских
картах, то есть кратчайшего пути между двумя точками, проходящего по
воде. Однако при нахождении единственного кратчайшего маршрута
возникают различные проблемы.

Во-первых, кратчайший путь не всегда является действительно
оптимальным, с точки зрения пользователя. Например, на пути могут быть
каналы, проплыть через которые возможно только за большую плату, что
сделает такой маршрут менее привлекательным, чем какой-то другой,
более длинный маршрут. В каких-то местах может быть нежелательно
плавать по политическим соображениям, также у капитана корабля могут
личные предпочтения. Формализовать всё множество критериев,
описывающих оптимальность маршрута, едва ли представляется возможным.

Во-вторых, иногда может возникнуть ситуация, при которой
воспользоваться кратчайшим маршрутом не представляется возможным.
Например, где-то могут проходить военные учения, из-за чего движение
судов в таких местах будет запрещено. Информация, заложенная в карту,
по которой строился маршрут, могла устареть, и какая-то река или канал
могли полностью высохнуть. В таком случае пользователю хочется иметь
альтернативный маршрут.

В-третьих, иногда нужно направить большое количество кораблей из одной
точки в другую. Если сотни кораблей пойдут одним маршрутом, то они
могут потратить очень много времени в очереди, чтобы проплыть по
какому-нибудь каналу. Если же корабли пустить разными маршрутами, то
суммарные временные затраты могут быть существенно уменьшены, несмотря
на то что часть кораблей поплывёт не по кратчайшему пути.

Таким образом, возникает задача поиска семейств маршрутов между двумя
точками. При этом маршруты должны быть в некотором смысле оптимальны.
Например, логично требовать, чтобы маршруты были несильно длиннее
кратчайшего пути и попарно непохожи. Непохожие маршруты должны
отличаться способом обхода существенных препятствий. Сама по себе
данная задача не приводит к конечному результату, а лишь осуществляет
поддержу для принятия решения. Окончательное решение принимается
пользователем в голове, поэтому основное требование к задачам
поддержки принятия решения состоит в том, что они должны решаться
практически моментально, в режиме реального времени, чтобы не сбивать
пользователя с мыслей.

Для задачи поиска нескольких маршрутов (multipath planning) также
известны некоторые решения. Например, существуют различные алгоритмы
решения задачи $K$-shortest paths, состоящей в поиске первых $K$ путей
в графе по возрастанию длины~\cite{eppstein1998finding,
yen1971finding}. Однако нетрудно понять, что обычно такие пути будут
иметь много общих рёбер и проходить через одни и те же водоёмы. Также
известны и другие алгоритмы multipath planning~\cite{lim2005shortest,
dial1971probabilistic, mafast}. Например,
алгоритм~\cite{lim2005shortest} находит пути, которые имеют как можно
меньше общих рёбер, однако при его применении к имеющейся задаче
получаются похожие маршруты. Это связано с тем, что если есть, скажем,
два маршрута, один из которых проходит на километр южнее другого, то
они, как правило, хоть и почти не имеют общих рёбер, по сути являются
очень похожими, поскольку обходят все препятствия с одной стороны
(пусть и на разном расстоянии).

В первой главе приведён подробный обзор предметной области и
существующих алгоритмов, формализована постановка задачи.
Во второй главе описано теоретическое решение поставленной задачи,
рассмотрены вопросы предобработки исходных данных и поиска семейств
маршрутов. В третьей главе рассмотрены вопросы эффективной реализации
предложенного решения и приведены основные результаты работы.

\FloatBarrier


%-*-coding: utf-8-*-
\chapter{Обзор предметной области}

\FloatBarrier
\section{Планирование путей}

\emph{Планирование путей (англ. path planning, motion planning)} ---
область computer science, решающая задачу поиска пути из одной точки в
другую, удовлетворяющего некоторым ограничениям. В основе
планирования путей лежат такие науки, как вычислительная геометрия
и теория графов. В данной работе рассматривается планирование
маршрутов по воде по всему миру. В этом случае суша представляет из
себя полигональные препятствия. Как известно (TODO: сослаться), любой
кратчайший путь между двумя вершинами при наличии полигональных препятствий
представляет собой ломаную, вершины которой --- вершины полигонов. В
дальнейшем в данной работе слова «\emph{маршрут}» и «\emph{путь}»
будут использованы взаимозаменяемо.

\emph{Граф видимости (англ. visibility graph)} --- граф, вершины
которого являются вершинами полигональных препятствий, а ребро между
вершинами $u$ и $v$ принадлежит графу тогда и только тогда, когда
отрезок $uv$ не пересекает ни одно препятствие.

Таким образом, любой кратчайший путь между двумя точками является
кратчайшим путём в графе видимости и может быть найден, например, с
помощью алгоритма Дейкстры. Однако граф видимости, построенный по
контурам суши со всего мира, взятым с мелкомасштабной карты, содержит
слишком много рёбер, из-за чего поиск может выполняться довольно
длительное время. В связи с этим возникает другая задача: найти
маршрут, близкий к кратчайшему. Для этой задачи необязательно строить
граф видимости, достаточно ограничиться навигационным графом.

\emph{Навигационный граф} --- граф, построенный по полигональным
препятствиям, вершины которого соответствуют некоторым точкам в
пространстве, а ребро между вершинами $u$ и $v$ может принадлежать графу только
тогда, когда отрезок $uv$ не пересекает ни одно препятствие.

\FloatBarrier
\section{Планирование мультипутей}

\emph{Планирование мультипутей (англ. multipath planning)} ---
область computer science, решающая задачу поиска нескольких путей из
одной точки в другую, таких что каждый путь удовлетворяет некоторым
ограничениям, и при этом на множеством путей также наложены
ограничения. Например, задача поиска $k$ кратчайших путей через
полигональные препятствия.

При поиске единственного маршрута (например, кратчайшего по какой-либо
метрике) могут возникнуть следующие проблемы:
\begin{enumerate}
    \item Критерии оптимальности не всегда очевидны и формализуемы.
      Чаще всего интересует кратчайший маршрут, но иногда может быть
      нужен, например, самый экономный маршрут, который не обязан
      совпадать с кратчайшим. Также могут быть и другие критерии, в
      том числе составные, которые едва ли представляется возможным
      описать какой-либо метрикой. Например, вряд ли возможно
      формализовать политические факторы или личные предпочтения пользователя.
    \item Ненадёжность. Возможна ситуация, когда находится
      единственный маршрут, но воспользоваться им не представляется
      возможным. Например, в каком-то месте может быть временный
      запрет или просто слишком мелко. В таком случае требуется
      запасной вариант.
    \item Повышенная загруженность. Если приложение, находящее ровно
      один маршрут, набирает популярность, это может привести к тому,
      что некоторыми маршрутами будут очень активно пользоваться, что
      приведёт к повышенной загруженности. Например, одна компания
      может направить большое число судов из одной точки в другую.
      Если все они пойдут через один и тот же канал, то могут потерять
      крайне много времени.
\end{enumerate}

Известны различные алгоритмы поиска нескольких маршрутов, более
подробный обзор которых будет приведён во второй главе.

\FloatBarrier
\section{Постановка задачи}

В данной работе требуется исследовать вопросы планирования
мультипутей на морских картах. Для этого прежде всего требуется
наложить ограничения на множество находимых маршрутов, тем самым
определив понятие семейства оптимальных маршрутов. Следующий этап
состоит в исследовании имеющихся алгоритмов планирования мультипутей
для выяснения их применимости к поиску маршрутов по воде. Затем
требуется разработать и реализовать новый алгоритм, находящий
семейство оптимальных маршрутов. Необходимо, чтобы алгоритм работал в
режиме реального времени, затрачивая меньше секунды на запрос.

\FloatBarrier
\section{Семейство оптимальных маршрутов}

При поиске нескольких маршрутов в первую очередь возникает вопрос о
том, каким условиям должно удовлетворять результирующее множество
маршрутов. Разумеется, маршруты не должны быть похожи. TODO

\FloatBarrier


\startconclusionpage

Нормально

\FloatBarrier


%\startappendices
%\label{appendix}
%\input{appendix}

\bibliographystyle{sty/utf8gost705u}
\bibliography{thesis}

\end{document}

